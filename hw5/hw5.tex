\documentclass[12pt]{article}
\usepackage{amsmath}
\usepackage{amsfonts}
\usepackage{tikz}
\usepackage{pgfplots}
\begin{document}
\title{Electrical Engineering 102, Homework 4}
\date{November 8th, 2018}
\author{Michael Wu\\UID: 404751542}
\maketitle

\section*{Problem 1}

\paragraph{a)}

The real component will contain the sine term and the imaginary component will not have a negative
sign and contain the cosine term. Thus we have the following.
\begin{align*}
    \mathbb{R}(c_k)&=-\mathbb{R}(c_{-k})\\
    \mathbb{I}(c_k)&=\mathbb{I}(c_{-k})\\
    c_k^*&=-c_{-k}\\
    |c_k|&=|c_{-k}|\\
    \angle c_k&=\pi-\angle c_{-k}
\end{align*}

\paragraph{b)}

\[x(t)=1\pm4\cos(2\pi t)\]
We can deduce this result by first noting that the frequency is \(1\) Hz, giving us \(\omega = 2\pi\).
Next since the DC component is 1, there must be a constant 1 added. So in the Fourier series expansion
of \(x(t)\) we have \(c_0=1\). Also, the other term must be a cosine since the signal is even and there is
only one positive frequency component. Finally from Parseval's theorem, we know that the total power is 9,
so there must be coefficients \(c_{-1}^2+c_0^2+c_1^2=9\). We can make the cosine term and satisfy this
equation by letting \(c_1=c_{-1}=\pm 2\), which corresponds to the term \(\pm 4\cos(2\pi t)\).

\paragraph{c)}

Assume that \(T>b-a\). Then the Fourier transform is an integral from \(a\) to \(b\), since the function is zero everywhere
else. Similarly, the Fourier series coefficients are integrals over one period, which is the same as \(a\) to \(b\) since
the function is zero everywhere else within the period. Thus we have that
\[c_k=\frac{1}{T}Y\left(jk\frac{2\pi}{T}\right)\]

\section*{Problem 2}

\paragraph{a)}

\begin{enumerate}
    \item a, d, e
    \item f
    \item c, e
    \item a, b
    \item e
    \item f
    \item d
    \item b
\end{enumerate}

\paragraph{b)}

\begin{enumerate}
    \item This is true. Convolution in the time domain corresponds to multiplication in the frequency domain. Thus the Fourier transform of the convolution must be an even and real function
    multiplied by an imaginary and odd function. This yields an odd and imaginary function, and taking the inverse Fourier transform of this would result in a real and odd function. Hence, convolution
    of a real and even signal and a real and odd signal is a real and odd signal.
    \item This is true. Let our signal be \(x(t)\) and let its Fourier transform be \(F(x(t))=X(j\omega)\). From time scaling we know that \(F(x(-t))=X(-j\omega)\). Then we have
    \[F(x(t)*x(-t))=X(j\omega)X(-j\omega)\]
    which produces an even function. An even Fourier transform corresponds to an even input signal, thus we know that \(x(t)*x(-t)\) is even.
\end{enumerate}

\paragraph{c)}

\begin{enumerate}
    \item If \(x(t)=x^*(-t)\), we know that \(\mathbb{R}(x(t))=\mathbb{R}(x(-t))\) and \(\mathbb{I}(x(t))=-\mathbb{I}(x(-t))\). So the real component of \(x(t)\) is even and the imaginary component of \(x(t)\)
    is odd. The Fourier transform of the real component is a real and even signal, while the Fourier transform of the imaginary component is a real and odd signal. From the linearity of the Fourier transform, we can
    add these two components together and get the Fourier transform for \(x(t)\). Thus \(X(j\omega)\) must be real because its components are real.
    \item We know that \(X(j\omega)=X_e(j\omega)+X_o(j\omega)\). Additionally, \(X_e(j\omega)\) must be real and even since \(x(t)\) is real. \(X_o(j\omega)\) must be imaginary and odd since \(x(t)\) is real.
    Therefore we know that the entirety of the real component of \(X(j\omega)\) comes from \(X_e(j\omega)\), so
    \[X_e(j\omega)=\mathbb{R}(X(j\omega))\]
    and the entirety of the imaginary component of \(X(j\omega)\) comes from \(X_o(j\omega)\), so
    \[X_o(j\omega)=j\mathbb{I}(X(j\omega))\]
\end{enumerate}

\section*{Problem 3}

\paragraph{a)}

We know that \(X(j\omega)\) is real and even. Furthermore, the integral
\[\int_{-\infty}^\infty X(j\omega)\, d\omega\]
is \(2\pi\) times the inverse Fourier transform of \(X(j\omega)\) at \(t=0\). As \(x(0)=1\), we get \(2\pi\) when evaluating
the inverse Fourier transform. Since the given integral is over the positive side of the domain and the Fourier transform is even,
we obtain a final result of \(\pi\).

\paragraph{b)}

This is the integral \(\int_{-\infty}^\infty x(t)\,dt\). Looking at the area under \(x(t)\), this evaluates to \(5\).

\paragraph{c)}

The Fourier transform is real and even. Thus either \(\angle X(j\omega)=0\) or \(\angle X(j\omega)=\pi\), depending on if it is positive or negative. We cannot
say more without explicitly finding \(X(j\omega)\) to see where it crosses the \(x\) axis.

\paragraph{d)}

This is \(2\pi\) times the inverse Fourier transform of \(X(j\omega)\) evaluated at \(t=-1\). Because \(x(-1)=1\), this integral is \(2\pi\).

\paragraph{e)}

Rewriting the expression yields
\begin{align*}
    F^{-1}(\mathbb{R}(e^{-j3\omega}X(j\omega)))&=F^{-1}(\cos(-3\omega)X(j\omega))\\
    &=F^{-1}\left(\frac{e^{-j3\omega} + e^{j3\omega}}{2}X(j\omega)\right)\\
    &=\frac{1}{2}x(t-3)+\frac{1}{2}x(t+3)
\end{align*}
which looks like the plot below.
\begin{center}
    \begin{tikzpicture}[scale=0.8]
        \begin{axis}[ymin=-0.99,ymax=0.99,xmax=6.99,xmin=-6.99,axis lines = middle]
            \addplot[color=black,mark=none] coordinates {
                (-7,0)
                (-6,0)
                (-5,0.5)
                (-1,0.5)
                (0,0)
                (1,0.5)
                (5,0.5)
                (6,0)
                (7,0)
            };
        \end{axis}
    \end{tikzpicture}
\end{center}

\section*{Problem 4}

\paragraph{a)}

\begin{enumerate}
    \item We can rewrite the function as follows
    \[x_1(t)=\text{rect}\left(\frac{-t-3}{2}\right)(e^{10\pi t} + e^{-10\pi t})\]
    The Fourier transform of the rect component is \(2e^{j3\omega}\text{sinc}\left(\frac{\omega}{\pi}\right)\). The exponential terms in the time domain modulate this Fourier transform, yielding
    \[F(x_1(t))=2e^{j3(\omega-10\pi)}\text{sinc}\left(\frac{\omega-10\pi}{\pi}\right)+2e^{j3(\omega+10\pi)}\text{sinc}\left(\frac{\omega+10\pi}{\pi}\right)\]
    \item We can rewrite the function as follows
    \[x_2(t)=e^2e^{j3t}e^{-2(-t+1)}u(-t+1)\]
    Then the last two terms gives the Fourier transform \(e^{-j\omega}\frac{1}{2-j\omega}\). The remaining exponential terms result in modulation and multiplication by a constant, yielding
    \[F(x_2(t))=\frac{e^{2-j(\omega-3)}}{2-j(\omega-3)}\]
    \item We have the following result.
    \begin{align*}
        F(x_3(t))&=\int_{-1}^1 e^{-j\omega t}\,dt + \frac{1}{2}\int_{-1}^1 e^{-j(\omega-\pi)t}\,dt + \frac{1}{2}\int_{-1}^1 e^{-j(\omega+\pi)t}\,dt\\
        &=-\frac{e^{-j\omega}-e^{j\omega}}{j\omega} - \frac{e^{-j(\omega-\pi)}-e^{j(\omega-\pi)}}{2j(\omega-\pi)} - \frac{e^{-j(\omega+\pi)}-e^{j(\omega+\pi)}}{2j(\omega+\pi)}\\
        &=2\frac{\sin(\omega)}{\omega}+\frac{\sin(\omega-\pi)}{\omega-\pi}+ \frac{\sin(\omega+\pi)}{\omega+\pi}\\
        &=2\text{sinc}\left(\frac{\omega}{\pi}\right) + \text{sinc}\left(\frac{\omega}{\pi}-1\right) + \text{sinc}\left(\frac{\omega}{\pi}+1\right)
    \end{align*}
    \item We have the following result.
    \begin{align*}
        F(x_4(t))&=\int_{0}^\infty te^{-t}e^{-j\omega t}\,dt\\
        &=\int_{0}^\infty te^{(-1-j\omega) t}\,dt\\
        &=\left.e^{(-1-j\omega) t}\left(\frac{t}{-1-j\omega}-\frac{1}{(-1-j\omega)^2}\right)\right|_0^\infty\\
        &=\frac{1}{(-1-j\omega)^2}
    \end{align*}
\end{enumerate}

\paragraph{b)}

\begin{align*}
    F^{-1}(X(j\omega))&=\frac{1}{2\pi}\left(\int_{-3}^{-2} e^{-j\frac{\omega}{2}} e^{j\omega t}\,d\omega + \int_{-2}^2 \frac{1}{2}e^{-j\frac{\omega}{2}} e^{j\omega t}\,d\omega
    + \int_2^3 e^{-j\frac{\omega}{2}} e^{j\omega t}\,d\omega\right)\\
    &=\frac{1}{2\pi}\left(\int_{-3}^{-2} e^{j\omega \left(t-\frac{1}{2}\right)}\,d\omega + \int_{-2}^2 \frac{1}{2}e^{j\omega \left(t-\frac{1}{2}\right)}\,d\omega
    + \int_2^3 e^{j\omega \left(t-\frac{1}{2}\right)}\,d\omega\right)\\
    &=\frac{1}{2\pi}\left(\left.\frac{e^{j\omega \left(t-\frac{1}{2}\right)}}{j\left(t-\frac{1}{2}\right)}\right|_{-3}^{-2}
    + \frac{1}{2}\left.\frac{e^{j\omega \left(t-\frac{1}{2}\right)}}{j\left(t-\frac{1}{2}\right)}\right|_{-2}^2
    + \left.\frac{e^{j\omega \left(t-\frac{1}{2}\right)}}{j\left(t-\frac{1}{2}\right)}\right|_2^3\right)\\
    &=\frac{1}{2\pi j\left(t-\frac{1}{2}\right)}\left(-e^{-j3\left(t-\frac{1}{2}\right)}+\frac{1}{2}e^{-j2\left(t-\frac{1}{2}\right)}-\frac{1}{2}e^{j2\left(t-\frac{1}{2}\right)}
    +e^{j3\left(t-\frac{1}{2}\right)}\right)\\
    &=\frac{\sin\left(3\left(t-\frac{1}{2}\right)\right)-\frac{1}{2}\sin\left(2\left(t-\frac{1}{2}\right)\right)}{\pi\left(t-\frac{1}{2}\right)}\\
    &=\frac{3}{\pi}\text{sinc}\left(\frac{3}{\pi}\left(t-\frac{1}{2}\right)\right)-\frac{1}{\pi}\text{sinc}\left(\frac{2}{\pi}\left(t-\frac{1}{2}\right)\right)
\end{align*}

\paragraph{c)}

\begin{enumerate}
    \item Based on duality we have that
    \[F_1(j\omega)=\frac{1}{2}\text{rect}\left(\frac{\omega}{4\pi}\right)\]
    Additionally, using duality and modulation we have that
    \[F_2(j\omega)=\frac{1}{2}\left(\text{rect}\left(\frac{\omega-3\pi}{2\pi}\right)+\text{rect}\left(\frac{\omega+3\pi}{2\pi}\right)\right)\]
    Applying the convolution theorem gives us
    \[F(j\omega)=\frac{1}{4}\text{rect}\left(\frac{\omega}{4\pi}\right)\left(\text{rect}\left(\frac{\omega-3\pi}{2\pi}\right)+\text{rect}\left(\frac{\omega+3\pi}{2\pi}\right)\right)\]
    This is zero everywhere except \(\omega=-2\pi\) and \(\omega=2\pi\), since the outer rect goes from \(-2\pi\) to \(2\pi\), while the other two rects go from \(-4\pi\) to \(-2\pi\) and
    \(2\pi\) to \(4\pi\). So we can also write
    \[F(j\omega)=\begin{cases}
        \frac{1}{4} & \omega=\pm 2\pi\\
        0 & \text{else}
    \end{cases}\]
    \item Since the inverse Fourier transform is an integral of a function that is zero everywhere except at a finite number of points, it must be zero. So we have \(f(t)=0\).
\end{enumerate}

\end{document}
