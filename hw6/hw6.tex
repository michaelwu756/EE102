\documentclass[12pt]{article}
\usepackage{amsmath}
\begin{document}
\title{Electrical Engineering 102, Homework 6}
\date{November 29th, 2018}
\author{Michael Wu\\UID: 404751542}
\maketitle

\section*{Problem 1}

\paragraph{a)}

This can be rewritten as follows.
\[\int_{-\infty}^\infty 2\text{sinc}^2(t)\cos^2(\pi t) - \text{sinc}^2(t)\, dt\]
Let \(f(t)=\text{sinc}(t)\cos(\pi t)\) and let
\[F(j\omega)=\frac{\text{rect}\left(\frac{\omega-\pi}{2\pi}\right)+\text{rect}\left(\frac{\omega+\pi}{2\pi}\right)}{2}\]
be its Fourier transform. Based on the properties of \(\text{rect}(t)\), we can rewrite the Fourier transform as
\[F(j\omega)=\frac{1}{2}\text{rect}\left(\frac{\omega}{4\pi}\right)\]
We also know that the Fourier transform of \(\text{sinc}(t)\) is \(\text{rect}\left(\frac{\omega}{2\pi}\right)\).
Then using Parseval's theorem this integral is equivalent to
\begin{align*}
    \int_{-\infty}^\infty \text{sinc}^2(t)\cos(2\pi t)\, dt
    &=\frac{1}{\pi}\int_{-\infty}^{\infty} |F(j\omega)|^2\,d\omega - \frac{1}{2\pi} \int_{-\infty}^\infty \left|\text{rect}\left(\frac{\omega}{2\pi}\right)\right|^2\,d\omega\\
    &=\frac{1}{\pi}\int_{-\infty}^{\infty}\frac{1}{4}\text{rect}\left(\frac{\omega}{4\pi}\right)\,d\omega - 1\\
    &=1-1\\
    &=0
\end{align*}

\paragraph{b)}

Based on duality we have
\[F[F[f(t)]]=2\pi f(-t)\]
Then applying the fourier transform four times should yield \(4\pi^2 f(t)\). Thus we have that
\[y(t)=4\pi^2x(t)\]

\section*{Problem 2}

\paragraph{a)}



\paragraph{b)}

\paragraph{c)}

\section*{Problem 3}

\paragraph{a)}

\paragraph{b)}

\paragraph{c)}

\paragraph{d)}

\section*{Problem 4}

\paragraph{a)}

\paragraph{b)}

\section*{Problem 5}

\paragraph{a)}

\paragraph{b)}

\paragraph{c)}

\end{document}