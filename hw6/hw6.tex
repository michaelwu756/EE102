\documentclass[12pt]{article}
\usepackage{pgfplots}
\usepackage{tikz}
\usepackage{amsmath}
\begin{document}
\title{Electrical Engineering 102, Homework 6}
\date{November 29th, 2018}
\author{Michael Wu\\UID: 404751542}
\maketitle

\pgfmathdeclarefunction{gauss}{2}{%
  \pgfmathparse{1/(#2*sqrt(2*pi))*exp(-((x-#1)^2)/(2*#2^2))}%
}

\section*{Problem 1}

\paragraph{a)}

This can be rewritten as follows.
\[\int_{-\infty}^\infty 2\text{sinc}^2(t)\cos^2(\pi t) - \text{sinc}^2(t)\, dt\]
Let \(f(t)=\text{sinc}(t)\cos(\pi t)\) and let
\[F(j\omega)=\frac{\text{rect}\left(\frac{\omega-\pi}{2\pi}\right)+\text{rect}\left(\frac{\omega+\pi}{2\pi}\right)}{2}\]
be its Fourier transform. Based on the properties of \(\text{rect}(t)\), we can rewrite the Fourier transform as
\[F(j\omega)=\frac{1}{2}\text{rect}\left(\frac{\omega}{4\pi}\right)\]
We also know that the Fourier transform of \(\text{sinc}(t)\) is \(\text{rect}\left(\frac{\omega}{2\pi}\right)\).
Then using Parseval's theorem this integral is equivalent to
\begin{align*}
    \int_{-\infty}^\infty \text{sinc}^2(t)\cos(2\pi t)\, dt
    &=\frac{1}{\pi}\int_{-\infty}^{\infty} |F(j\omega)|^2\,d\omega - \frac{1}{2\pi} \int_{-\infty}^\infty \left|\text{rect}\left(\frac{\omega}{2\pi}\right)\right|^2\,d\omega\\
    &=\frac{1}{\pi}\int_{-\infty}^{\infty}\frac{1}{4}\text{rect}\left(\frac{\omega}{4\pi}\right)\,d\omega - 1\\
    &=1-1\\
    &=0
\end{align*}

\paragraph{b)}

Based on duality we have
\[\mathcal{F}[\mathcal{F}[f(t)]]=2\pi f(-t)\]
Then applying the Fourier transform four times should yield \(4\pi^2 f(t)\). Thus we have that
\[y(t)=4\pi^2x(t)\]

\section*{Problem 2}

\paragraph{a)}

\begin{enumerate}
    \item We know that \(y(t)=h(t)*x(t)\). Taking the Fourier transform of the system yields the following.
    \begin{align*}
        \mathcal{F}[y^{\prime\prime}(t)+6y^\prime(t)+8y(t)]&=\mathcal{F}[3x(t)]\\
        \mathcal{F}[h(t)*x(t)](-\omega^2+j6\omega+8)&=3\mathcal{F}[x(t)]\\
        \mathcal{F}[h(t)]\mathcal{F}[x(t)](-\omega^2+j6\omega+8)&=3\mathcal{F}[x(t)]\\
        \mathcal{F}[h(t)]&=\frac{3}{-\omega^2+j6\omega+8}\\
        \mathcal{F}[h(t)]&=\frac{3}{2(2+j\omega)}-\frac{3}{2(4+j\omega)}
    \end{align*}
    So our frequency response is
    \[H(j\omega)=\frac{3}{2(2+j\omega)}-\frac{3}{2(4+j\omega)}\]
    \item We know that \(y(t)=h_2(t)*y_1(t)\). Then plugging in the given \(y(t)\) and \(y_1(t)\) and taking the Fourier transform yields the following.
    \begin{align*}
        \mathcal{F}[(4e^{-t}-4e^{-4t})u(t)]&=\mathcal{F}[h_2(t)*(2e^{-t}u(t))]\\
        \frac{4}{1+j\omega}-\frac{4}{4+j\omega}&=\mathcal{F}[h_2(t)]\left(\frac{2}{1+j\omega}\right)\\
        \mathcal{F}[h_2(t)]&=2-\frac{2+j2\omega}{4+j\omega}\\
        \mathcal{F}[h_2(t)]&=\frac{6}{4+j\omega}
    \end{align*}
    So our frequency response is
    \[H_2(j\omega)=\frac{6}{4+j\omega}\]
    We also know that \(H(j\omega)=H_1(j\omega)H_2(j\omega)\), so we can solve for \(H_1(j\omega)\) as follows.
    \begin{align*}
        H_1(j\omega)\frac{6}{4+j\omega}&=\frac{2}{3(2+j\omega)}-\frac{2}{3(4+j\omega)}\\
        H_1(j\omega)\frac{6}{4+j\omega}&=\frac{3}{-\omega^2+j6\omega+8}\\
        H_1(j\omega)&=\frac{1}{2(2+j\omega)}
    \end{align*}
    \item Using the inverse Fourier transform rule for \(e^{-at}u(t)\) we obtain the following impulse responses.
    \begin{align*}
        h(t)&=\frac{3}{2}(e^{-2t}-e^{-4t})u(t)\\
        h_1(t)&=\frac{1}{2}e^{-2t}u(t)\\
        h_2(t)&=6e^{-4t}u(t)
    \end{align*}
\end{enumerate}

\paragraph{b)}

No, \(y(t)\) cannot have frequency components different than those of \(x(t)\). This is because the system is LTI, so it can be expressed
as a convolution integral \(h(t)*x(t)\). This corresponds to multiplication in the frequency domain. Then since \(X(j\omega)\) is zero
for \(|\omega|>\omega_0\), \(Y(j\omega)\) must also be zero for \(|\omega|>\omega_0\). If we process \(x(t)\) through a non-LTI system then
yes, \(y(t)\) can have frequency components different than those of \(x(t)\). Consider the system \(y(t)=x(t)+\cos(2\omega_0 t)\) which has
\(Y(j\omega)\neq 0\) for \(|2\omega_0|>\omega_0\).

\paragraph{c)}

The Fourier transform of \(x(t)\) can be found by applying the modulation rule twice, while the Fourier transform of \(h_1(t)\) can be found using
the Fourier transform of \(\text{sinc}(at)\) and modulation. Then we can calculate the result of the first system by doing the following.
\begin{align*}
    y_1(t)&=x(t)*h_1(t)\\
    Y_1(j\omega)&=X(j\omega)H_1(j\omega)\\
    Y_1(j\omega)&=\frac{\pi}{2}(\delta(\omega-7\pi)+\delta(\omega-\pi)+\delta(\omega+\pi)+\delta(\omega+7\pi))\times\\
    &\qquad\left(\text{rect}\left(\frac{\omega-\pi}{\pi}\right)+\text{rect}\left(\frac{\omega+\pi}{\pi}\right)\right)\\
    Y_1(j\omega)&=\frac{\pi}{2}(\delta(\omega-\pi)+\delta(\omega+\pi))\\
    y_1(t)&=\frac{\cos(\pi t)}{2}
\end{align*}
The Fourier transform of \(h_2(t)\) can be found using the Fourier transform of \(\text{sinc}(at)\). Then we can calculate the result of the second system by doing the following.
\begin{align*}
    y_2(t)&=x(t)*h_2(t)\\
    Y_2(j\omega)&=X(j\omega)H_2(j\omega)\\
    Y_2(j\omega)&=\frac{\pi}{2}(\delta(\omega-7\pi)+\delta(\omega-\pi)+\delta(\omega+\pi)+\delta(\omega+7\pi))\text{rect}\left(\frac{\omega}{4\pi}\right)\\
    Y_2(j\omega)&=\frac{\pi}{2}(\delta(\omega-\pi)+\delta(\omega+\pi))\\
    y_2(t)&=\frac{\cos(\pi t)}{2}
\end{align*}
If we are given an input-output pair for an unknown LTI system then we cannot always identify this system uniquely. For example, if we are given the input \(x(t)=\cos(3\pi t)\cos(4\pi t)\)
and the output \(y(t)=\frac{\cos(\pi t)}{2}\), the results above indicate that our LTI system could be characterized by either \(h_1(t)\) or \(h_2(t)\) as impulse responses.

\section*{Problem 3}

\paragraph{a)}

We should set \(\alpha=-1\). This way the lower frequencies cancel out the frequencies that pass through the low pass filter. This has the effect of making the high pass filter multiply the strength
of the signal by \(-1\) as well as filtering, so the frequency response of the high pass filter has a phase of \(\pi\).

\paragraph{b)}

Ideal filters are non-realizable systems because the impulse response of an ideal filter is a sinc function which always has some nonzero component as time goes to positive or negative infinity.
In other words, the impulse response is non-causal and never dies out. So when processing a real signal, we cannot look in the future to take into account the non-causality of the impulse response. We also
cannot process a signal for infinitely long, as that would require infinite resources and the universe only has finite resources.

\paragraph{c)}

In order to have unity at \(\omega=0\) we must set \(k=\beta\). The magnitude of the frequency response is equal to
\begin{align*}
    \left|\frac{k}{\beta+j\omega}\right|&=\left|\frac{1}{\beta^2+\omega^2}(k\beta-jk\omega)\right|\\
    &=\frac{k\sqrt{\beta^2+\omega^2}}{\beta^2+\omega^2}\\
    &=\frac{k}{\sqrt{\beta^2+\omega^2}}
\end{align*}
So we need to solve the following equation in order to set the cutoff frequency.
\begin{align*}
    \frac{k}{\sqrt{\beta^2+(2\pi)^2}}&=\frac{1}{\sqrt{2}}\\
    k&=\sqrt{\frac{\beta^2+(2\pi)^2}{2}}\\
    k^2&=\frac{\beta^2+(2\pi)^2}{2}\\
    2k^2&=\beta^2+(2\pi)^2\\
    2k^2&=k^2+(2\pi)^2\\
    k^2&=(2\pi)^2\\
    k&=2\pi
\end{align*}
Thus we obtain the result \(k=2\pi\) and \(\beta=2\pi\).

\paragraph{d)}

The system \(\alpha x(t)\) is equivalent to \((\alpha\delta(t))*x(t)\). The frequency response of \(\alpha\delta(t)\) is equal to \(\alpha\), thus by linearity we obtain the frequency response of the high pass filter as
\[H(j\omega)=\frac{2\pi}{2\pi+j\omega}-1\]
This is a non ideal high pass filter that shifts the signal through a phase of \(\pi\), since it tends to minimize low frequency signals while it multiplies the by \(-1\). High frequency signals come through at almost full
strength, except they are opposite in phase.

\section*{Problem 4}

\paragraph{a)}

\begin{enumerate}
    \item The output of the modulator has the frequency response
    \[X(j\omega)=\frac{M(j(\omega-\omega_c))+M(j(\omega+\omega_c))}{2}\]
    which is shown below.
    \begin{center}
        \begin{tikzpicture}
            \begin{axis}[
                no markers, domain=-10:10, samples=500,
                axis line style=thick,
                axis lines=middle, enlargelimits=true, xlabel=\(\omega\), ylabel=\(X(j\omega)\),
                every axis y label/.style={at=(current axis.above origin),anchor=south},
                every axis x label/.style={at=(current axis.right of origin),anchor=west},
                height=3cm, width=11cm,
                xtick={-9,-7,-5,5,7,9},
                xticklabels={\empty, \(-\omega_c\), \empty, \empty, \(\omega_c\), \empty},
                ytick=\empty,
                ymin=0, ymax=1,
                enlargelimits=false, clip=false, axis on top,
                grid = none
            ]
                \addplot [very thick,black!50!black] {gauss(-7,0.6)};
                \addplot [very thick,black!50!black] {gauss(7,0.6)};
            \end{axis}
        \end{tikzpicture}
    \end{center}
    \item The output of the bandpass would have a frequency response that looks like the following graph.
    \begin{center}
        \begin{tikzpicture}
            \begin{axis}[
                no markers, domain=-10:10, samples=500,
                axis line style=thick,
                axis lines=middle, enlargelimits=true, xlabel=\(\omega\), ylabel=\(Y(j\omega)\),
                every axis y label/.style={at=(current axis.above origin),anchor=south},
                every axis x label/.style={at=(current axis.right of origin),anchor=west},
                height=3cm, width=11cm,
                xtick={-9,-7,-5,5,7,9},
                xticklabels={\empty, \(-\omega_c\), \empty, \empty, \(\omega_c\), \empty},
                ytick=\empty,
                ymin=0, ymax=1,
                enlargelimits=false, clip=false, axis on top,
                grid = none
            ]
                \addplot [very thick,black!50!black] {(\x>-7)*gauss(-7,0.6)};
                \addplot [very thick,black!50!black] {(\x<7)*gauss(7,0.6)};
            \end{axis}
        \end{tikzpicture}
    \end{center}
    \item The output of the demodulator would have a frequency response that looks like the following graph.
    \begin{center}
        \begin{tikzpicture}
            \begin{axis}[
                no markers, domain=-10:10, samples=500,
                axis line style=thick,
                axis lines=middle, enlargelimits=true, xlabel=\(\omega\), ylabel=\(Z(j\omega)\),
                every axis y label/.style={at=(current axis.above origin),anchor=south},
                every axis x label/.style={at=(current axis.right of origin),anchor=west},
                height=3cm, width=11cm,
                xtick={-8,-7,-6,-1,0,1,6,7,8},
                xticklabels={\empty, \(-2\omega_c\), \empty, \empty, \(0\), \empty, \empty, \(2\omega_c\), \empty},
                ytick=\empty,
                ymin=0, ymax=4,
                enlargelimits=false, clip=false, axis on top,
                grid = none
            ]
                \addplot [very thick,black!50!black] {gauss(0,0.3)};
                \addplot [very thick,black!50!black] {(\x>-7)*gauss(-7,0.3)};
                \addplot [very thick,black!50!black] {(\x<7)*gauss(7,0.3)};
            \end{axis}
        \end{tikzpicture}
    \end{center}
    Yes the system would recover \(m(t)\), but at half amplitude as the original signal. The ideal low pass filter would attenuate the frequency components centered around \(\pm 2\omega_c\).
    It would also multiply by \(2\) to correct for the reduction in signal strength. But the original signal \(m(t)\) undergoes two modulations by \(\cos(\omega_c t)\) which reduces the magnitude of the frequency
    components by a factor of \(4\). This results in the output being half the amplitude as the original.
\end{enumerate}

\paragraph{b)}

The sampled signal is equal to
\[z(t)=d_{\frac{2\pi}{\omega_c}}(t)y(t)\]
So its frequency response can be found by taking the Fourier transform of this expression which gives
\begin{align*}
    \mathcal{F}[z(t)]&=\mathcal{F}\left[d_{\frac{2\pi}{\omega_c}}(t)\right]*\mathcal{F}[y(t)]\\
    &=\omega_c d_{\omega_c}(\omega)*Y(j\omega)
\end{align*}
Then using the flip and drag method we obtain a frequency response of the following form.
\begin{center}
    \begin{tikzpicture}
        \begin{axis}[
            no markers, domain=-10:10, samples=500,
            axis line style=thick,
            axis lines=middle, enlargelimits=true, xlabel=\(\omega\), ylabel=\(Z(j\omega)\),
            every axis y label/.style={at=(current axis.above origin),anchor=south},
            every axis x label/.style={at=(current axis.right of origin),anchor=west},
            height=3cm, width=11cm,
            xtick={-9,-7,-5,-2,0,2,5,7,9},
            xticklabels={\empty, \(-\omega_c\), \empty, \empty, \(0\), \empty,\empty, \(\omega_c\), \empty},
            ytick=\empty,
            ymin=0, ymax=1,
            enlargelimits=false, clip=false, axis on top,
            grid = none
        ]
            \addplot [very thick,black!50!black] {gauss(0,0.6)};
            \addplot [very thick,black!50!black] {gauss(-7,0.6)};
            \addplot [very thick,black!50!black] {gauss(7,0.6)};
        \end{axis}
    \end{tikzpicture}
\end{center}
in which the frequency response of \(\frac{m(t)}{2}\) is repeated at every multiple of \(\omega_c\) in the frequency domain. Taking into account the scaling factor \(\omega_c\), the low pass
filter would then recover the original signal multiplied by \(\omega_c\) in magnitude. This has a frequency response as shown below, which is the same shape as our original signal.
\begin{center}
    \begin{tikzpicture}
        \begin{axis}[
            no markers, domain=-10:10, samples=500,
            axis line style=thick,
            axis lines=middle, enlargelimits=true, xlabel=\(\omega\), ylabel=\(M(j\omega)\),
            every axis y label/.style={at=(current axis.above origin),anchor=south},
            every axis x label/.style={at=(current axis.right of origin),anchor=west},
            height=3cm, width=11cm,
            xtick={-2,0,2},
            xticklabels={\empty, \(0\), \empty},
            ytick=\empty,
            ymin=0, ymax=1,
            enlargelimits=false, clip=false, axis on top,
            grid = none
        ]
            \addplot [very thick,black!50!black] {gauss(0,0.6)};
        \end{axis}
    \end{tikzpicture}
\end{center}

\section*{Problem 5}

\paragraph{a)}

The maximum frequency in hertz is \(B\) Hz. Thus the Nyquist rate is the same as the original signal if the signal is time shifted, which is \(2B\) Hz.

\paragraph{b)}

This corresponds to convolution with
\[\frac{1}{2}(\delta(\omega-2\pi B)+\delta(\omega+2\pi B))\]
in the frequency domain. Thus the maximum frequency in Hz increases to \(2B\) Hz. So the Nyquist rate is double the original signal's Nyquist rate.
Thus it is \(4B\) Hz.

\paragraph{c)}

Time scaling by \(\frac{1}{2}\) stretches out the signal, corresponds to lowering the bandwidth by a factor of \(\frac{1}{2}\). Thus the maximum bandwidth comes from
the original signal, so the Nyquist rate is the same as the original signal which is \(2B\) Hz.

\end{document}
